%%%%%%%%%%%%%%%%%%%%%%%%%%%%%%%%%%%%%%%%%
% Twenty Seconds Resume/CV
% LaTeX Template
% Version 1.0 (14/7/16)
%
% Original author:
% Carmine Spagnuolo (cspagnuolo@unisa.it) with major modifications by 
% Vel (vel@LaTeXTemplates.com) and Harsh (harsh.gadgil@gmail.com)
%
% License:
% The MIT License (see included LICENSE file)
%
%%%%%%%%%%%%%%%%%%%%%%%%%%%%%%%%%%%%%%%%%

%----------------------------------------------------------------------------------------
%	PACKAGES AND OTHER DOCUMENT CONFIGURATIONS
%----------------------------------------------------------------------------------------

\documentclass[letterpaper]{twentysecondcv} % a4paper for A4
\newcommand{\tightlist}{}
% Command for printing skill overview bubbles
\newcommand\skills{ 
~
    \smartdiagram[bubble diagram]{
        \textbf{Astronomer~~},
        \textbf{~Bayesian~}\\\textbf{Statistics},
        \textbf{~Big Data~}\\\textbf{\& HPC},
        \textbf{~Observer~},
        \textbf{Numerical}\\
        \textbf{Techniques}
    }
}


% Programming skill bars
\programming{Python}{{Shell $\textbullet$ SQL $\textbullet$ Matlab $\textbullet$ \large \LaTeX}}{{C $\textbullet$ C++  $\textbullet$ R $\textbullet$ Ruby $\textbullet$ IDL $\textbullet$ html}}

% Projects text
\education{
     \textbf{Ph.D., Astronomy} \\
 University of Hertfordshire, UK \\
 2015 - 2019 \\
 Passed viva w/ minor corrections\\

 \textbf{MPhys, Physics} \\
 Durham University, UK \\
 2010 - 2014 \\
 2:1 with Honours\\

}

\affiliations{
 Fellow of the Royal Astronomical Society, \textit{FRAS}\\
 Member of the European Astronomical Society, \textit{EAS}\\
}

%----------------------------------------------------------------------------------------
%	 PERSONAL INFORMATION
%----------------------------------------------------------------------------------------
% If you don't need one or more of the below, just remove the content leaving the command, e.g. \cvnumberphone{}

\cvname{Shaun C Read} % Your name
\cvjobtitle{Postdoc} % Job
% title/career

\cvlinkedin{}
\cvgithub{philastrophist}
\cvnumberphone{}
\cvsite{shaun.science/} % Personal website
\cvmail{shaun.c.read@gmail.com} % Email address

%----------------------------------------------------------------------------------------

\begin{document}

\makeprofile % Print the sidebar
 
%----------------------------------------------------------------------------------------
%	 EXPERIENCE
%----------------------------------------------------------------------------------------

\section{Summary}
     I am a postdoctoral researcher specialising in Bayesian statistical analysis on big data, working on the weak-lensing colour-gradient bias with Euclid.
My main interests are reverberation mapping and the interface between star-forming galaxies and AGN. 
I have worked with a diverse range of data including the latest releases from the LOFAR, SDSS, and H-ATLAS surveys and the Horizon-AGN simulations. 
My latest work combines the use of novel statistical Bayesian analysis with these large datasets in order to facilitate effective exploitation of the 
next generation of surveys.


\subsection{Research Interests}
\begin{itemize}
     \item \textbf{Star-formation}: LOFAR, FIR, empirical relations, FIRC, MagPhys, SFG-AGN interface.
 \item \textbf{Reverberation mapping}: High redshift, photometric techniques, $t_{lag}-L_{5100}$, selection
biases.
 \item \textbf{Big data \& Bayesian analysis}: Large surveys, advanced Bayesian statistical inference, bias mitigation.
\end{itemize}

\section{Experience}

\begin{twenty} % Environment for a list with descriptions \twentyitem{<dates>}{<title>}{<location>}{<description>}
     \twentyitem{Oct 2019}{Present}{Postdoc}{Osservatorio Astronomico di Roma - INAF}{Galaxy shape measures in Euclid}{\begin{itemize}
\tightlist
\item
  Quanitfying the colour-gradient bias in Euclid weak-lensing
  measurements
\item
  Generation of realistic galaxy catalogues
\item
  Hubble image reduction
\end{itemize}}
 \twentyitem{Oct 2015}{Oct 2019}{Ph.D.}{University of Hertfordshire}{Supervisor: Dr Daniel J.B. Smith}{Thesis: Measuring the Physical Properties of Distant Galaxies and Black
Holes in the Era of Surveys\begin{itemize}
\tightlist
\item
  Studying the relation between the star-formation rate and radio
  luminosity of galaxies.
\item
  Using new photometric time-series techniques to estimate quasar
  black-hole masses with reverberation mapping.
\item
  Innovating new Bayesian methods to infer complete distributions from
  incomplete, noisy data in order to mitigate observational bias and
  explore large datasets.
\end{itemize}}
 \twentyitem{Jun 2016}{}{Observing}{William Herschel Telescope, La Palma}{}{}
 \twentyitem{Jan 2016}{Present}{Programming teaching assistant \& tutor}{University of Hertfordshire, UK}{}{\begin{itemize}
\tightlist
\item
  Taught students Python and Matlab for scientific programming courses.
\item
  Ran code review sessions for post-graduates and Ph.D.~students.
\item
  Lead programming lectures and demonstrations.
\end{itemize}}
 \twentyitem{Nov 2016}{Mar 2017}{`Physics of stars' demonstrator}{University of Hertfordshire, UK}{}{\begin{itemize}
\tightlist
\item
  Assisted students at the Bayfordbury teaching observatory.
\item
  Instructed in the use of 16-inch telescopes and the reduction of data.
\item
  Projects included PNe imaging and constructing open cluster
  HR-diagrams.
\end{itemize}}
 \twentyitem{Jul 2014}{Jul 2015}{Insight Analyst}{Linkdex, UK}{Processing big data from raw consumer search patterns to an explanative
format suitable for client business strategies.}{\begin{itemize}
\tightlist
\item
  Big data processing with Python \& sci-kit learn
\item
  Communication with the backend team
\item
  API design, visualisation, and automation development.
\end{itemize}}
\end{twenty}

\begin{finalpages}
\subsection{Other Experience}
\begin{twentyfull}
 \twentyitemfull{Jun 2013}{Aug 2013}{Summer Student}{National Physical Laboratory, UK}{Supervisor: Dr Alastair Sinclair}{\begin{itemize}
\tightlist
\item
  Worked with the Time \& Frequency Team.
\item
  Analysed Gaussian beam quality for the strontium ion optical clock
  group.
\item
  Developed analytical Matlab code and the optical bench setup required.
\end{itemize}}
\end{twentyfull}

\section{Presentations}
\begin{twentyfull}
     \twentyitemfull{April 2018}{}{European Week of Astronomy and Space Science}{European Astronomical Society, \emph{EAS}}{University of Liverpool, UK}{poster}
 \twentyitemfull{July 2017}{}{National Astronomy Meeting}{Royal Astronomical Society, \emph{RAS}}{University of Hull, UK}{contributed talk}
 \twentyitemfull{June 2016}{}{National Astronomy Meeting}{Royal Astronomical Society, \emph{RAS}}{University of Nottingham, UK}{contributed talk, poster}
 \twentyitemfull{May 2016}{}{The Cosmic FIR Landscape with H-ATLAS}{H-ATLAS consortium}{University of Lisbon, Portugal}{contributed talk}
\end{twentyfull}

\section{Publications}
\subsection{Published}
\begin{itemize}
     \item \href{http://shaun.science/publication/2020-01-00-Galaxy-morphological-classification-in-deep-wide-surveys-via-unsupervised-machine-learning}{\textit{Galaxy morphological classification in deep-wide surveys via
unsupervised machine learning}\\{\small Martin, G.; Kaviraj, S.; Hocking, A.; \textbf{Read, S.C.}; Geach, J.E. -- 2020MNRAS.491.1408M}}
 \item \href{http://shaun.science/publication/2019-12-00-The-Performance-of-Photometric-Reverberation-Mapping-at-High-Redshift-and-the-Reliability-of-Damped-Random-Walk-Models}{\textit{The Performance of Photometric Reverberation Mapping at High Redshift
and the Reliability of Damped Random Walk Models}\\{\small \textbf{Read, S.C.}; Smith, D.J.B.; Jarvis, M.J.; Gürkan, G. -- 2019MNRAS.tmp.3203R}}
 \item \href{http://shaun.science/publication/2019-11-00-A-LOFAR-IRAS-cross-match-study}{\textit{A LOFAR-IRAS cross-match study: the far-infrared radio correlation and
the 150 MHz luminosity as a star-formation rate tracer}\\{\small Wang, L.; Gao, F.; Duncan, K.J.; Williams, W.L.; Rowan-Robinson, M.;
Sabater, J.; Shimwell, T.W.; Bonato, M.; Calistro-Rivera, G.; Chyży,
K.T.; Farrah, D.; Gürkan, G.; Hardcastle, M.J.; McCheyne, I.; Prandoni,
I.; \textbf{Read, S.C.}; Röttgering, H.J.A.; Smith, D.J.B. -- 2019A\&A\ldots631A.109W}}
 \item \href{http://shaun.science/publication/2018-11-00-The-Far-Infrared-Radio-Correlation-at-low-radio-frequency-with-LOFAR-H-ATLAS}{\textit{The Far-Infrared Radio Correlation at low radio frequency with
LOFAR/H-ATLAS}\\{\small \textbf{Read, S.C.}; Smith, D.J.B.; Gürkan, G.; Hardcastle, M.J.;
Williams, W.L.; Best, P.N.; Brinks, E.; Calistro-Rivera, G.; ChyŻy,
K.T.; Duncan, K.; Dunne, L.; Jarvis, M.J.; Morabito, L.K.; Prandoni, I.;
Röttgering, H.J.A.; Sabater, J.; Viaene, S. -- 2018MNRAS.480.5625R}}
 \item \href{http://shaun.science/publication/2016-10-00-LOFAR-H-ATLAS}{\textit{LOFAR/H-ATLAS: a deep low-frequency survey of the Herschel-ATLAS North
Galactic Pole field}\\{\small Hardcastle, M.J.; Gürkan, G.; van Weeren, R.J.; Williams, W.L.; Best,
P.N.; de Gasperin, F.; Rafferty, D.A.; \textbf{Read, S.C.}; Sabater, J.;
Shimwell, T.W.; Smith, D.J.B.; Tasse, C.; Bourne, N.; Brienza, M.;
Brüggen, M.; Brunetti, G.; Chyży, K.T.; Conway, J.; Dunne, L.; Eales,
S.A.; Maddox, S.J.; Jarvis, M.J.; Mahony, E.K.; Morganti, R.; Prandoni,
I.; Röttgering, H.J.A.; Valiante, E.; White, G.J. -- 2016MNRAS.462.1910H}}
 \item \href{http://shaun.science/publication/2016-10-00-The-Astropy-Problem}{\textit{The Astropy Problem}\\{\small Muna, D.; Alexander, M.; Allen, A.; Ashley, R.; Asmus, D.; Azzollini,
R.; Bannister, M.; Beaton, R.; Benson, A.; Berriman, G.B.; Bilicki, M.;
Boyce, P.; Bridge, J.; Cami, J.; Cangi, E.; Chen, X.; Christiny, N.;
Clark, C.; Collins, M.; Comparat, J.; Cook, N.; Croton, D.; Delberth
Davids, I.; Depagne, É.; Donor, J.; dos Santos, L.A.; Douglas, S.; Du,
A.; \ldots; \textbf{Read, S.}; \ldots{} -- 2016arXiv161003159M}}
\end{itemize}

\subsection{Submitted and in preparation}
\begin{itemize}
     \item \textit{On the causes of the mass dependency of the star-formation rate -- radio
luminosity relation with LOFAR, Horizon-AGN, and CANDID}\\{\small \textbf{Read, S.}; Smith, D.; Gürkan, G.; Hardcastle, M.; et al. -- in prep.}
 \item \textit{A Markov Chain Monte Carlo approach for measurement of jet precession in
radio-loud active galactic nuclei}\\{\small Horton, M.; Hardcastle, M.; \textbf{Read, S.}; Krause, M. -- submitted to MNRAS}
 \item \textit{Bias and accretion rate dependency in the reverberation-mapped
lag-luminosity relation}\\{\small \textbf{Read, S.}; Smith, D.; et al. -- in prep.}
 \item \textit{Low mass stars and multiple systems in Gaia}\\{\small González-Egea, E.; Pinfield, D.; \textbf{Read, S.}; et al. -- in prep.}
\end{itemize}
\end{finalpages}
\end{document} 